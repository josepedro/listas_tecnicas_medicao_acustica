\chapter[Riscos Técnicos]{Riscos Técnicos}

O primeiro risco identificado foi o de não conseguir fazer tudo o que foi inicialmente proposto. Este tinha uma alta chance de acontecer, já que algumas propostas feitas pelo grupo eram um pouco incoerentes, como fazer uma análise e comparação dos erros capturados pela ferramenta utilizada. Para mitigar tal incerteza, foi sugerido, pela professora da disciplina, reduzir o escopo e focar no guia já citado e no funcionamento da ferramenta escolhida, que ainda é pouco conhecida.

Outro evento que poderia ter algumas consequências ruins para o projeto é o time da disciplina de Métodos de Desenvolvimento de Software (MDS), responsável pela produção do software em questão, não conseguir desenvolver os requisitos especificados ou o mesmo gerá-los tardiamente para o próposito deste trabalho. Para mitigar este efeito, o grupo decidiu implementar as histórias de usuário, caso o grupo de MDS, por algum motivo, não consiga fazê-los. Existe uma probabilidade maior de não ser possível a implementação a tempo, do que o fato de MDS não conseguir implementar, já que o time é bem dedicado e se esforça para cumprir os prazos propostos.

Um risco com baixíssima probabilidade de ocorrer e que foi mitigado nas primeiras semanas do projeto é o de o grupo não conseguir desenvolver uma aplicação na linguagem e $framework$ propostos ($Python$/$Django$). A mitigação deste se deu através da troca de experiência com alguns integrantes do time que já possuíam certa experiência com as tecnologias citadas. Estes integrantes foram tanto do grupo de Verificação e Validação de Sofware, quanto do de Gerência de Portfólios e Projetos de Software.

Também existia o risco do grupo não conseguir utilizar a ferramenta proposta (devido à dificuldades de instalação, configuração e a própria usabilidade da mesma). Este foi mitigado pela própria documentação do $Lettuce$, que proveu todas as funcionalidades básicas para a sua utilização. Além disso, a experiência com ferramentas como o $Cucumber$, e a escrita em $Python$ facilitaram muito a interação do grupo com o $Lettuce$.